\section*{Проектирование базы данных}
\addcontentsline{toc}{section}{Проектирование базы данных}

База данных является неотъемлемой частью любого приложения, которое выполняет
хранение и обеспечивает работу с информацией. Так что выбор правильной технологии
хранения данных является чрезвычайно важной задачей. 

На текущий момент существует две различные ветви развития систем управления
базами данных(СУБД): релационная и нереляционная.

Отличительной чертой реляционных баз данных является понятие отношения или таблицы.
Каждая сущность, хранимая в базе данных, должна представлять собой строку таблицы со 
строго заданным типизированным набором столбцов. Также реляционные СУБД гарантируют
выполнение так называемых свойств ACID к транзакционной системе, где под транзакцией
понимается последовательность команд, представляющая логическую единицу работы с данными.
Опишем свойства ACID: 

\begin{enumerate}
    \item Атомарность -- транзация либо будет выполнена целиком, либо
        не выполнена совсем;
    \item Согласованность -- после выполнения транзакции в базе данных
        находятся корректные значения;
    \item Изолированность -- на транзакцию не могут оказать влияния другие транзакции,
        выполняемые параллельно;
    \item Устойчивость -- если транзакция была завершена, то даже при сбое системы
        изменения будут зафиксированы.
\end{enumerate}

Данные свойства накладывают довольно серьезные ограничения на производительность, что
послужило поводом для появления нереляционных СУБД. Перед ними стояло требование
обеспечить хранение данных для высоконагруженных приложений. В противовес
свойствам ACID, нереляционные базы данных гарантируют выполнение свойств BASE
(Источник: What NoSQL is and what it is not.):

\begin{enumerate}
    \item Доступность -- каждый запрос будет выполнен;
    \item Гибкость -- состояние системы может меняться со временем даже без ввода новых данных;
    \item Согласованность в конечном счете -- данные могут быть несогласованны в некоторые
        моменты времени, но в итоге приходят в согласованное состояние.
\end{enumerate}

//Проверить точность формулировок в первоисточнике

Стоит отметить, что существует множество видов нереляционных баз данных, перечислим
основные:

\begin{enumerate}
    \item документоориентированные;
    \item графовые;
    \item ключ-значение;
\end{enumerate}

Графовые базы данных не дадут особого выигрыша из-за относительной простоты хранимых данных.

В произведенном далее сравнении все замечания, относящиеся к документоориентированным базам данных
в равной степени относятся и к базам данных вида ключ-значения, так что ниже речь будет идти
только о реляционных и документоориентированных СУБД.
