\section*{Тестирование}
\addcontentsline{toc}{section}{Тестирование}

Как уже отмечалось выше, потенциальная производительность данного приложения не является главной характеристикой
разрабатываемого приложения, так как даже потенциально у приложения не будет больше десяти пользователей,
подключенных одновременно. Это означает, что в данном разделе будет более целесообразно остановиться на
рассмотрении того пользовательского опыта, который сможет получить человек при использовании разработанного
приложения. Такое тестирование называется тестированием эргономичности или usability testing.

Итак, основная цель, которую будет приследовать пользователь при использовании приложения -- это создание
текстового файла со списком использованной литературы, который можно импортировать в LaTeX-документ.

Так как в проекте предусмотрено два различных и параллельных друг другу способа взаимодействия с приложением,
ниже будет рассмотрен каждый из них. Для удобства работа с приложением командной строки будет приведена
в листингах, а с веб-версией -- снимками экрана. Причем в листингах знаком ">" будет обозначаться введенная
bash-команда.

Первым делом необходимо создать кафедру, на которой работает пользователь. Для наглядности будет рассмотрена
кафедра <<Теоретическая информатика и компьютерные технологии>>.

Необходимые действия, которые нужно сделать в приложении командной строки, приведены в листинге ~\ref{addDepartment}.

\begin{lstlisting}[language=bash, caption = {Добавление кафедры}, captionpos=b, label={addDepartment}]
> cli department prototype
Open department.txt and fill prototype struct with correct data
> nano department.txt 
> cli department add
\end{lstlisting}

В файле department.txt будет находиться JSON-объект, для которого будет необходимо заполнить поле \texttt{title}
необходимым значением.