\section*{Тестирование}
\addcontentsline{toc}{section}{Тестирование}

Как уже отмечалось выше, потенциальная производительность данного приложения не является главной характеристикой
разрабатываемого приложения, так как даже потенциально у приложения не будет больше десяти пользователей,
подключенных одновременно. Это означает, что в данном разделе будет более целесообразно остановиться на
рассмотрении того пользовательского опыта, который сможет получить человек при использовании разработанного
приложения. Такое тестирование называется тестированием эргономичности или usability testing.

Итак, основная цель, которую будет приследовать пользователь при использовании приложения -- это создание
текстового файла со списком использованной литературы, который можно импортировать в LaTeX-документ.

Так как в проекте предусмотрено два различных и параллельных друг другу способа взаимодействия с приложением,
ниже будет рассмотрен каждый из них. Для удобства работа с приложением командной строки будет приведена
в листингах, а с веб-версией -- снимками экрана. Причем в листингах знаком ">" будет обозначаться введенная
bash-команда.

Для наглядности будет рассмотрен курс <<Базы данных>>, преподаваемый на кафедре <<Теоретическая информатика и компьютерные технологии>>
Вишняковым Игорем Эдуардовичем.

Первым делом необходимо создать кафедру, на которой работает пользователь.

Необходимые действия, которые нужно сделать в приложении командной строки, приведены в листинге ~\ref{cli_add_department}.

\begin{lstlisting}[language=bash, caption = {Добавление кафедры}, captionpos=b, label={cli_add_department}]
> cli department prototype
Open department.txt and fill prototype struct with correct data
> nano department.txt 
> cli department add
\end{lstlisting}

В файле department.txt будет находиться JSON-объект, для которого будет необходимо заполнить поле \texttt{title}
необходимым значением.

В случае работы с веб-приложением нужно выбрать необходимую таблицу, после чего в нее автоматически загрузится прототип
искомой JSON-структуры, который следует модифицировать и отправить на сервер кнопкой \texttt{Upload}, расположенной справа снизу
от поля ввода текста. Иллюстрацию можно видеть на рисунке ~\ref{web_add_department}.

\begin{figure}[h!]
	\center{\includegraphics[width=1\linewidth]{web_add_department.png}}
	\caption{Добавление новой кафедры в веб-приложении}
	\label{web_add_department}
\end{figure}

Следующее действие -- добавление лектора к только что введенной кафедре. Последовательность действий
ничем не отличается от предыдущего шага и ее можно увидеть в листинге ~\ref{cli_add_lecturer} и рисунке
~\ref{web_add_lecturer}.

\begin{lstlisting}[language=bash, caption = {Добавление лектора}, captionpos=b, label={cli_add_lecturer}]
> cli lecturer prototype
Open lecturer.txt and fill prototype struct with correct data
> nano lecturer.txt 
> cli lecturer add
\end{lstlisting}

\begin{figure}[h!]
	\center{\includegraphics[width=1\linewidth]{web_add_lecturer.png}}
	\caption{Добавление нового лектора в веб-приложении}
	\label{web_add_lecturer}
\end{figure}

На шаге добавления курса становится видно преимущество веб-версии по сравнению с приложением командной строки.
Оно позволяет в двух разных окнах вводить информацию и делать запросы к базе данных. Иллюстрацию можно видеть
на рисунке ~\ref{web_add_course}. В этом примере можно держать перед глазами список добавленных лекторов чтобы,
например, не забыть дату рождения, и одновременно вводить данные об учебном курсе.

\begin{figure}[h!]
	\center{\includegraphics[width=1\linewidth]{web_add_course.png}}
	\caption{Добавление учебного курса в веб-приложении}
	\label{web_add_course}
\end{figure}

В то время как в приложении командной строки такой возможности нет. Как вариант, можно отдельно запросить список
лекторов командой \texttt{cli lector get}, чтобы открыть его в отдельном окне терминала или текстовом редакторе, но этот вариант 
явно проигрывает по удобству. Так что последовательность команд остается аналогичной и приведена в листинге ~\ref{cli_add_course}.

\begin{lstlisting}[language=bash, caption = {Добавление лектора}, captionpos=b, label={cli_add_course}]
> cli course prototype
Open course.txt and fill prototype struct with correct data
> nano course.txt 
> cli course add
\end{lstlisting}

Этап добавления нового учебного курса ничем не отличается от всех предыдущих, так что имеет смысл перейти к добавлению книг в приложение.
Как уже было сказано выше, для этого предусмотрен модуль, позволяющий использовать сервис Google Books для поиска информации о
книгах. Но возможно и добавление книг вручную, если нужного учебника не нашлось в базе сервиса. 
Для начала рассмотрим эту функцию в приложении командной строки из листинга ~\ref{cli_search}.

\begin{lstlisting}[language=bash, caption = {Поиск книг по запросу SQL в сервисе Google Books в приложении командной строки.}, captionpos=b, label={cli_search}]
> cli search --request="SQL"
Open searchResults.txt, view results, remove wrong items and fix incorrect data.
> nano searchResults.txt 
> cli book prototype
Open book.txt and fill prototype struct with correct data
> nano book.txt
> cli book add
\end{lstlisting}

Другими словами, необходимо сначала выполнить поиск, получить результат в файл, после чего скопировать нужную книгу в
\texttt{book.txt} и добавить ее командой \texttt{cli book add}.

Схожий принцип, но более удобный, используется и в веб-приложении, что можно видеть на рисунке ~\ref{web_search}.

\begin{figure}[h!]
	\center{\includegraphics[width=1\linewidth]{web_search.png}}
	\caption{Поиск книг по запросу SQL в сервисе Google Books в веб-приложении.}
	\label{web_search}
\end{figure}

Добавление книги в список литературы также является набором тех же самых действий, так что имеет смысл перейти сразу к дополнительной функции:
миграции списков литературы. По своей сути, эта функция является абсолютно тем же заполнением нужной JSON-структуры, так что будет
удобно привести только вариант веб-приложения. Его можно видеть на рисунке ~\ref{web_migrate}

\begin{figure}[h!]
	\center{\includegraphics[width=1\linewidth]{web_migrate.png}}
	\caption{Демонстрация миграции списка литературы в веб-приложении.}
	\label{web_migrate}
\end{figure}

Важной деталью, которую необходимо отметить и зафиксировать, является то, что необходимо создать заранее тот список литературы, в который
будет производиться миграция.

Наконец стало возможным перейти к основному и самому важному пункту -- генерации самого отчета.