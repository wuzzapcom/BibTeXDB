 \documentclass[a4paper,14pt]{report}
\usepackage[utf8x]{inputenc} % Включаем поддержку UTF8
\usepackage{extsizes} % Включаем поддержку 14 размера
\usepackage[T2A]{fontenc} % Поддержка русских букв
\usepackage{ucs}
\usepackage[russian, english]{babel}  % Включаем пакет для поддержки русского языка
\usepackage[left=3cm,right=1cm,top=2cm,bottom=2cm]{geometry} % Размер отступов
\usepackage{sectsty}
\usepackage[pdftex]{graphicx}
\usepackage{listings}
\usepackage{xcolor}
\usepackage{amsmath}
\usepackage{amsfonts}
\usepackage{enumitem}
\usepackage{cite}
\usepackage{multirow} % улучшенное форматирование таблиц; источник - https://habr.com/post/144648/
\usepackage{indentfirst} % Красная строка
\usepackage{tocloft} %https://tex.stackexchange.com/questions/316626/list-of-listing-space-before-title
\usepackage[nottoc]{tocbibind}
% \usepackage{flafter}% помещает флоат(изображение) ПОСЛЕ первой ссылки на него; источник - http://mydebianblog.blogspot.ru/2013/03/amorua-advanced-floats.html
% Отображение списка литературы в содержании как section, а не chapter
\makeatletter
\renewenvironment{thebibliography}[1]
     {\section*{\bibname}% <-- this line was changed from \chapter* to \section*
      \@mkboth{\MakeUppercase\bibname}{\MakeUppercase\bibname}%
      \list{\@biblabel{\@arabic\c@enumiv}}%
           {\settowidth\labelwidth{\@biblabel{#1}}%
            \leftmargin\labelwidth
            \advance\leftmargin\labelsep
            \@openbib@code
            \usecounter{enumiv}%
            \let\p@enumiv\@empty
            \renewcommand\theenumiv{\@arabic\c@enumiv}}%
      \sloppy
      \clubpenalty4000
      \@clubpenalty \clubpenalty
      \widowpenalty4000%
      \sfcode`\.\@m}
     {\def\@noitemerr
       {\@latex@warning{Empty `thebibliography' environment}}%
      \endlist}
\makeatother

\setlist{nolistsep} % убрать лишние интервалы в списках
\linespread{1.3} % полуторный интервал
% \renewcommand{\rmdefault}{ftm} % Times New Roman
\lstdefinestyle{sharpc}{language=bash, frame=lr, rulecolor=\color{blue!80!black}}
\sloppy


\author{Vladimir Lapatin}
\DeclareGraphicsExtensions{.pdf,.png,.jpg}
\graphicspath{{pics/}}



\begin{document}

\renewcommand{\figurename}{Рисунок} % Замена Figure на Рисунок в подписи изображений
\renewcommand{\lstlistingname}{Листинг}

%Титульная страница
\begin{titlepage}	% начало титульной страницы

\begin{center}	
%\\[0.5cm]
\LargeМосковский государственный технический университет имени Н.Э. Баумана\\
\large {(МГТУ им. Н.Э. Баумана)}\\[0.4cm] 
\rule[+3mm]{7.5cm}{0.80mm}
\begin{flushleft}
\begin{tabbing}
ММММММММММ \= \kill
\large{ФАКУЛЬТЕТ} \> \large{\textit{ИНФОРМАТИКА И СИСТЕМЫ УПРАВЛЕНИЯ}} \\	\large{КАФЕДРА} \> \large{\textit{ТЕОРЕТИЧЕСКАЯ ИНФОРМАТИКА}} \\
 \> \large{\textit{И КОМПЬЮТЕРНЫЕ ТЕХНОЛОГИИ}}
\\[1.0cm]
\end{tabbing}
\end{flushleft}
\end{center}


\begin{center}


\Huge{Курсовая работа}\\
\LARGE{<<Приложение хранения библиографических ссылок>>.}
\\[0.5cm]
\LARGE{По курсу <<Базы данных>>.}
\\[2.0cm]




 \rule[+0mm]{6.2cm}{0.0mm}Выполнил:  \rule[+0mm]{1.3cm}{0.0mm} Лапатин В.В.\\
 \rule[+0mm]{6.2cm}{0.0mm}Проверил:  \rule[+0mm]{1.1cm}{0.0mm} Дубанов А.В.\\ [3.5cm]



\large{Москва 2018}
\end{center}
\thispagestyle{empty} % не нумеровать страницу
\end{titlepage} % конец титульной страницы
\clearpage
% -----------------
% \setcounter{page}{2}
% Содержание
\setlength{\cftbeforetoctitleskip}{-2em} %https://tex.stackexchange.com/questions/46724/remove-vertical-space-before-table-of-contents-title
\renewcommand{\contentsname}{Содержание}
\tableofcontents
\thispagestyle{empty}
\clearpage
% -----------------
\setcounter{page}{3}
% Введение
\section*{Введение}
\addcontentsline{toc}{section}{Введение}

Данная курсовая работа посвящена базам данных. В современном мире каждому приложению необходимо хранить и обрабатывать
различную информацию, что делает чрезвычайно важным понимание разработчиком принципов работы с информацией. Для этого
необходимо знать о различных средствах хранения информации и иметь представление о механизмах их работы. Также важным является
умение разрабатывать серверные приложения, которые осуществляют взаимодействия с базами данных.

Таким образом, цель данной курсовой работы заключается в изучении различных средств хранения информации, создании базы
данных и разработке приложений, осуществляющих взаимодействие пользователя с хранимой информацией.

В первой главе рассматривается разработка интерфейса и определяется основная архитектура будущего приложения. На этом этапе
будет более подробно рассмотрена поставленная задача, так как без этого невозможно грамотно разработать архитектуру приложения.

Во второй главе производится выбор будущей системы управления базами данных, а также осуществляется
проектирование базы данных.

В третьей главе описывается структура разработанного серверного приложения.

В заключительной, четвертой, главе, производится тестирование пользовательского опыта и приводится подробное
описание принципов работы с разработанным приложением.
\clearpage
% -----------------

% Глава 1
\section*{Обзор}
\addcontentsline{toc}{section}{1 Обзор}

Для любой программы интерфейс является одной из наиболее важных составляющих.
Именно он определяет, как приложение будет взаимодействовать с другими
программами и своими пользователями. Таким образом, можно ввести следующую
классификацию интерфейсов:

\setlist{nolistsep}
\begin{enumerate}
    \item Интерфейсы программирования приложений.
    \item Графические.
    \item Интерфейсы командной строки.
\end{enumerate}

Несмотря на то, что эти интерфейсы имеют между собой мало общего, к ним 
предъявляется ряд общих требований:

\begin{enumerate}
    \item Функциональность -- интерфейс должен отвечать всем требованиям пользователя и соответствовать его задачам.
    \item Логичность -- интерфейс должен быть логичным и запоминающимся, чтобы взаимодействие пользователя
        с программой было как можно более простым и удобным.
    \item Защищенность -- интерфейс должен быть спроектирован таким образом, чтобы у пользователя не было
        возможности совершить ошибку.
\end{enumerate}

После определения общих требований к любому интерфейсу, стоит рассмотреть перечисленные выше
интерфейсы отдельно.

\subsection*{Интерфейс программирования приложений}
\addcontentsline{toc}{subsection}{1.1 Интерфейс программирования приложений}

К интерфейсам программирования приложений можно причислить любой интерфейс, который предназначен для
использования разрабатываемой программы внешними приложениями. Обычно, в зависимости от
используемых технологий, это набор классов, функций или методов, которые используются
внешними программами. В случае данного приложения в качестве интерфейса взаимодействия
было решено использовать веб-технологии. Это означает, что приложение будет взаимодействовать
с любыми своими пользователями через протокол HTTP\cite{RFC2616}.
Таким образом, программный интерфейс приложения будет представлять из себя набор HTTP-методов.

В протоколе HTTP есть несколько видов методов, из которых приложением будут использоваться следующие:

\begin{enumerate}
    \item GET -- это методы, которые запрашивают данные, они не предназначены для их записи.
    \item POST -- это методы, которые используются и для записи данных, и для их получения.
\end{enumerate}

Говоря об HTTP интерфейсах, стоит отметить, что есть несколько различных подходов к их проектированию. 
Одним из наиболее общепринятых подходов является так называемый REST.
Это можно расшифровать как Represental State Transfer\cite{REST}. Данная архитектура предлагает
наложить на приложение ряд следующих ограничений:

\begin{enumerate}
    \item Модель клиент-сервер -- означает, что вся логика должна выполняться на удаленном сервере,
    а клиентское приложение должно исключительно предоставлять и получать данные.
    \item Отсутствие состояния -- сервер получает из запроса всю необходимую информацию и не хранит
    никакую информацию о сессии клиентов.
    \item Кеширование -- сервер сохраняет наиболее частые ответы, что позволяет не выполнять
    лишние запросы к базе данных и соответствующие вычисления, а сразу вернуть результат,
    \item Единообразие интерфейса.
    \item Уровни абстракции -- сокрытие основного сервера за промежуточными. Например, без каких-либо
    изменений для пользователя можно внедрить между ним и сервером промежуточный сервер, который
    предназначен для хранения и возврата хэшированных данных пользователю. В случае отсутствия этих данных
    хэширующий сервер перенаправляет запрос исходному серверу.
\end{enumerate}

Такой подход позволяет добиться лучшей производительности за счет отсутствия состояний между
вызовами и кеширования, а архитектура становится более расширяемой из-за требований
единообразия и выделения уровней абстракции.

\subsection*{Графический интерфейс пользователя}
\addcontentsline{toc}{subsection}{1.2 Графический интерфейс пользователя}

Графический интерфейс пользователя -- это наиболее востребованные интерфейсы в современном мире, потому
что пользователям намного более удобно пользоваться интерфейсами, которые для взаимодействия
с пользователем могут использовать не только текст. 

И, пожалуй, самый популярный вид графических приложений -- это веб-сайты.
Они получили такую популярность из-за своей универсальности: пользователь
может открыть приложение на любом устройстве, на котором есть доступ в интернет
и веб-браузер, в то время как любое другое приложение потребует установки
на устройство пользователя. С другой стороны, веб-сайты очень удобны для
разработчиков. Намного более выгодно разработать одно веб-приложение, чем
создавать и поддерживать несколько программ для разных платформ.

Обычно разработку веб-приложения можно разделить на две части:
<<клиентская>> и <<серверная>>.

Клиентская -- это та часть приложения, которая исполняется в браузере пользователя.
Для написания веб-сайтов используются язык разметки HTML, язык описания стилей CSS и интерпретируемый
язык программирования JavaScript\cite{Web-book}.

Опишем роль каждого языка в веб-приложении:
\begin{enumerate}
    \item HTML -- это каркас всего приложения, который определяет, какие элементы будут использоваться и
    где они будут располагаться,
    \item CSS определяет, как будут выглядеть элементы приложения. Сюда входят, например, внешний вид кнопок
    или вид шрифта отображаемого текста,
    \item JavaScript -- это интерпретируемый язык программирования, основная область применения которого заключается
    в придании интерактивности веб-страницам. Например, динамическую загрузку данных в таблицу можно реализовать только
    через JavaScript. 
    
\end{enumerate}

Серверной обычно называют ту часть приложения, которая стоит за клиентской. В основном это
та программа, которая предоставляет данные, выполняет их хранение и агрегацию. Другими словами,
сервер реализует всю <<бизнес-логику>>, в то время как клиент нужен для предоставления
пользователям доступа к приложению. Более подробно об этом будет рассказано далее.

\subsection*{Интерфейс командной строки}
\addcontentsline{toc}{subsection}{1.3 Интерфейс командной строки}

Интерфейсы командной строки, в отличии от графических и программных интерфейсов, представляют
наименее популярную группу приложений. Но, несмотря на это, они являются
незаменимыми. Главное преимущество приложений командной строки заключается в том, что
ими могут пользоваться в равной степени эффективно человек, и программа. Таким образом,
интерфейсы командной строки являются сочетанием программных и графических интерфейсов:
с одной стороны ими с определенной степенью удобства может пользоваться человек, а с другой
стороны без каких-либо трудностей они могут использоваться для взаимодействия между
приложениями. Также стоит отметить, что приложениям командной строки не нужен какой-либо
графический интерфейс, только окно терминала.

\subsection*{Выбор СУБД}
\addcontentsline{toc}{subsection}{1.4 Выбор СУБД}

База данных является неотъемлемой частью любого приложения, которое выполняет
хранение и обеспечивает работу с информацией. Так что выбор правильной технологии
хранения данных является чрезвычайно важной задачей. 

На текущий момент существует две различные ветви развития систем управления
базами данных(СУБД)\cite{GettingStartedWithNoSQL}: релационная и нереляционная.

Отличительной чертой реляционных баз данных является понятие отношения или таблицы.
Каждая сущность, хранимая в базе данных, должна представлять собой строку таблицы со 
строго заданным типизированным набором столбцов. Также реляционные СУБД гарантируют
выполнение так называемых свойств ACID к транзакционной системе, где под транзакцией
понимается последовательность команд, представляющая логическую единицу работы с данными.
Опишем свойства ACID: 

\begin{enumerate}
	\item Атомарность -- транзация либо будет выполнена целиком, либо
	      не выполнена совсем.
	\item Согласованность -- после выполнения транзакции в базе данных
	      находятся корректные значения.
	\item Изолированность -- на транзакцию не могут оказать влияния другие транзакции,
	      выполняемые параллельно.
	\item Устойчивость -- если транзакция была завершена, то даже при сбое системы
	      изменения будут зафиксированы.
\end{enumerate}

Данные свойства накладывают довольно серьезные ограничения на производительность, что
послужило поводом для появления нереляционных СУБД. Их задачей было
обеспечить хранение данных в высоконагруженных приложениях. В противовес
свойствам ACID, нереляционные базы данных гарантируют выполнение свойств BASE:

\begin{enumerate}
	\item Доступность -- каждый запрос будет выполнен.
	\item Гибкость -- состояние системы может меняться со временем даже без ввода новых данных.
	\item Согласованность в конечном счете -- данные могут быть несогласованы в некоторые
	      моменты времени, но в итоге приходят в согласованное состояние.
\end{enumerate}

Стоит отметить, что существует множество видов нереляционных баз данных, перечислим
основные:

\begin{enumerate}
	\item Документоориентированные -- хранят данные в документах.
	\item Графовые -- хранят данные в виде графа.
	\item ключ-значение -- хранят данные в виде пар вида "ключ-значение".
\end{enumerate}

Так как данные, которые будут храниться в будущем приложении, достаточно просты по своей структуре,
никакого выигрыша от использования графовых баз данных получить не получится.

Базы данных вида ключ-значение также не очень хорошо подходят для поставленной задачи, так как их структура
является слишком простой, так что возникнут дополнительные трудности в работе с такой базой данных.

Таким образом, остаются документоориентированные базы данных. Было решено сравнить реляционные и нереляционные
базы данных на примере PostreSQL и MongoDB.

В основе документоориентированной базы данных лежит понятие документа. В общем случае в качестве формата
может быть множество различных стандартов: JSON, XML, YAML и так далее. В случае MongoDB для хранения
используется BSON\cite{MongoDBDocs} -- бинарное надмножество JSON. В данной СУБД документы группируются в так называемые
коллекции. В отличии от реляционной модели, коллекция не имеет какой-либо строгой структуры. То есть
в ней могут храниться абсолютно разные документы.

Данных подход имеет ряд своих преимуществ и недостатков. Перечислим положительные стороны:

\begin{enumerate}
	\item Гибкость -- так как коллекция не ограничена структурой хранимых данных, информация
	      в ней может несколько отличаться от записи к записи. Это может быть полезно, если
	      данные имеют в целом общий смысл, но в некоторых документах могут присутствовать
	      какие-то особые поля.
	\item Производительность -- так как не происходит никаких проверок, запись и чтение
	      работают существенно быстрее, чем в реляционном подходе.
\end{enumerate}

Недостатки и ограничения:

\begin{enumerate}
	\item Подобная организация плохо подходит для создание ссылок между объектами по, например,
	      первичному ключу. Это связано с описанным выше отсутствием проверок. Канонический
	      способ хранения данных -- это использование вложенных документов. Но этот способ
	      применим не везде из-за ограничения на размер документа в 16 МБ.
	\item Как было описано в пункте выше, довольно сложно производить нормализацию
	      данных, потому что все проверки необходимо производить не на уровне СУБД, а
	      на уровне приложения.
\end{enumerate}

Теперь стоит проанализировать данные, которые будут храниться в приложении.

Все данные имеют абсолютно строгую структуру. Если говорить о хранимых книгах, то для них есть набор полей, определенный
стандартом BibTeX, который обязан быть у каждой записи. Остальные данные, такие как списки литературы
и учебные курсы, также не имеют никакой вариативности. Таким образом, гибкость NoSQL подхода только 
добавит сложностей в связи с необходимостю ручной реализации множества проверок.

Конечная цель приложения -- генерировать списки литературы
для учебных курсов. И весьма логичным требованием будет то, что в любой момент времени приложение
должно генерировать корректные отчеты. Таким образом, для данной задачи больше подходят требования
ACID, чем BASE.

Как видно из рассуждений выше, несмотря на все свои преимущества, для поставленной задачи больше
подходит реляционная модель.

\clearpage
% -----------------

% Глава 2
\section*{Реализация}
\addcontentsline{toc}{section}{Реализация}

\subsection*{Модель сущность-связь}
\addcontentsline{toc}{subsection}{Модель сущность-связь}

Первым делом необходимо формализовать таблицы базы данных. Опишем нужные таблицы:

\begin{itemize}
	\item Textbook -- таблица с книгами в формате BibTeX. Обладает несколькими UNIQUE столбцами:
	      \begin{itemize}
	      	\item ident -- идентификатор книги, должен быть уникальным, так как именно
	      	      он используется для идентификации библиографической ссылки в стандарте LaTeX.
	      	\item isbn -- уникальный для каждой книги ключ, позволяет исключить дублирование.
	      \end{itemize}
	\item LiteratureList -- таблица, хранящая списки литературы. Имеет UNIQUE ограничение на
	      пару из ID курса, которому присвоен список и года этого курса. Это необходимо для
	      идентификации списка литературы.
	\item Literature -- так называемая таблица пересечения, необходимая для создания связи
	      многие-ко-многим между таблицами LiteratureList и Textbook.
	\item Course -- задает учебный курс. Стоит отметить, что один учебный курс может иметь
	      несколько списков литературы за разные года. Имеет ссылки на кафедру, к которой 
	      привязан и лектора, читающего курс. Обладает UNIQUE ограничением на тройку из
	      названия, кафедры и семестра, в котором читается курс.
	\item Department -- задает кафедру, имеет UNIQUE поле title, характеризующее
	      название кафедры.
	\item Lecturer -- хранит всех лекторов. Однозначно идентифицируется именем и датой рождения.
\end{itemize}

Все таблицы обладают суррогатными первичными ключами.

Стоит также отметить два важных решения, которые были приняты для всей базы данных.

Во-первых, каждая запись имеет поле timestamp -- временную метку добавления или изменения записи.
Это позволит иметь историю изменений. Также нужно уточнить, что данное поле имеет тип integer,
что является более общим решением, чем хранение метки во внутреннем формате СУБД.

Во-вторых, в базе данных будет отсутствовать возможность удаления записи. Для этого каждая
запись имеет флаг isDeleted. При удалении пользователем записи она будет помечаться, как удаленная.
У данного решения есть ряд преимуществ. Это позволяет обезопасить базу данных от ошибок пользователя
и от потенциального взлома системы. И в том, и в другом случае никому не удастся нанести
непоправимый ущерб данным. Причем блокировка будет осуществляться за счет того, что у учетной записи,
через которую пользователь будет взаимодействовать с базой данных, не будет прав на удаление из базы данных.
При этом все равно будет администратор, у которого данная возможность есть.

В итоге получаются диаграмма модели сущность-связь, показанная на рисунке ~\ref{ris:ermodel}.

\begin{figure}[h!]
	\center{\includegraphics[width=1\linewidth]{schema.png}}
	\caption{Диаграмма модели сущность-связь}
	\label{ris:ermodel}
\end{figure}

Стоит отдельно отметить минимальные кардинальные связи в этой базе данных. Для упрощения конфигурации серверной части предполагается 
запрет любым пользователям, кроме администратора, какого-либо удаления. Удаление предполагается каскадное, реализованное на серверной части
с помощью обновления флага isDeleted. Таким образом, везде неявно предполагается минимальная кардинальная связь единица.

В качестве альтернативного решения можно было бы использовать триггеры, которые вместо удаления будут
производить обновление поля. Но в таком случае пострадает переносимость. Дело в том, что не в каждой
SQL базе данных присутствуют гибкие настройки прав и будет уже не так легко заменить базу данных.
На уровне серверной части работа с базой данной реализована через стандартную библиотеку для работы с
SQL, что дает возможность будущей миграции на любую SQL базу данных. Более подробно реализация будет описана ниже.
В то время как предложенный подход не имеет такой завязки на настройку прав пользователей.

Выбор максимальных кардинальных связей основан исключительно на функциях таблиц, описанных выше и предполагаемой
логики. Объясним выбор этих связей:

\begin{itemize}
	\item Department-Lecturer -- предполагается, что преподаватель может числиться только на одной кафедре, а у кафедры 
	      может быть много преподавателей,
	\item Department-Course -- аналогично случаю Department-Lecturer,
	\item Lecturer-Course -- обычно в университетских программах при наличии нескольких преподавателей в учебном
	      курсе главным считается лектор, так как именно он определяет программу курса, а значит и его список литературы. 
	      Таким образом, для идентификации курса достаточно одного преподавателя -- лектора. С другой стороны, лектор может
	      вести несколько учебных курсов.
	\item Course-LiteratureList -- у каждого курса может быть несколько разных списков литературы за разные года, но
	      у каждого списка только один курс,
	\item LiteratureList-Literature-Textbook -- связь многие-ко-многим между LiteratureList и Textbook, так как
	      каждый список содержит множество книг, а учебник может быть использован в нескольких курсах.
\end{itemize}

\subsection*{Интерфейс программирования приложений}
\addcontentsline{toc}{subsection}{Интерфейс программирования приложений}

При рассмотрении интерфейсов программирования приложений была описана архитертура REST. Согласно
этой архитектуре каждый метод должен принимать всю необходимую информацию
для выполнения запроса. Таким образом, для каждого вида данных был реализован следующий набор методов:

\begin{itemize}
    \item add -- POST запрос, содержащий данные для добавления в теле,
    \item get -- GET запрос, получающий данные от приложения,
    \item prototype -- GET запрос, получающий прототип JSON-файла для искомой таблицы.
\end{itemize}

Также были реализованы две дополнительных команды: report и migrate.

Первая получает данные, необходимые для однозначной идентификации требуемого списка литературы
и генерирует для этого списка литературы файл в формате BibTeX. 

Команда migrate выполняет копирование списка литературы с одного года на другой. Это необходимо для того,
чтобы облегчить работу пользователя, ведь обычно учебные программы слабо меняются от года к году.

\subsection*{Интерфейс командной строки}
\addcontentsline{toc}{subsection}{Интерфейс командной строки}

Как уже было сказано выше, для интерфейсов командной строки крайне важна автоматизируемость.
Это делает невозможным использование так называемого интерактивного ввода. Другими словами,
любое действие пользователя должно совершаться за один вызов приложения. В результате
было решено использовать следующий подход к проектированию интерфейса: для каждого вида данных,
с которыми будет работать приложение, будет реализована собственная команда. И для каждой такой
команды будет реализован набор подкоманд, выполняющих необходимые действия. В общем случае
команду приложения командной строки можно описать в следующем виде:

\begin{itemize}
    \item prototype -- данный метод добавляет новый файл в файловую систему,
        который содержит в себе прототип для нужного вида данных. После чего пользователь
        должен заполнить этот прототип информацией, которую хочет добавить,
    \item add -- сохраняет данные в приложение,
    \item get -- позволяет получить данные из приложения, можно конфигурировать
        флагами.
\end{itemize}

Также отличительной чертой хорошего приложения командной строки является грамотно оформленная справка.
Рисунок ~\ref{ris:help_example} показывает результат выполнения команды help. 

\begin{figure}[h!]
    \center{\includegraphics[width=1\linewidth]{help}}
    \caption{Пример работы команды help.}
    \label{ris:help_example}
\end{figure}

Для каждой отдельной команды реализован
отдельный флаг --help, который показывает справку для конкретной команды, что можно видеть на рисунке 
~\ref{ris:help_book_example}. 

\begin{figure}[h!]
\center{\includegraphics[width=1\linewidth]{help_book}}
\caption{Пример работы флага help для команды book}
\label{ris:help_book_example}
\end{figure}

\subsection*{Графический интерфейс}
\addcontentsline{toc}{subsection}{Графический интерфейс}

Как уже было сказано выше, графический интерфейс реализован в виде веб-сайта. Приложение было написано на языке TypeScript с 
использованием фреймворка Bootstrap.
Выбор фреймворка обусловлен тем, что он предоставляет продвинутый набор стилей и HTML-элементов, что позволяет 
легко и быстро настроить внешний вид веб-приложения.
В связи с тем, что единственный современный и поддерживаемый язык для написания веб-сайтов - это JavaScript, то 
из альтернатив ему можно использовать только языки, компилируемые в JavaScript. В частности, к таким языкам можно 
отнести язык TypeScript компании Microsoft и язык Kotlin, разрабатываемый компанией JetBrains, который имеет 
отдельный плагин для компиляции в JS-код. К сожалению, Kotlin является очень молодым языком, который находится в 
активной разработке, так что было решено использовать TypeScript\cite{TypeScript}. Преимуществом данного языка является то, что 
он является надмножеством JavaScript. Это означает, что любой код JavaScript является абсолютно корректным. Также 
важно отметить, что TypeScript поддерживает важные аспекты объектно-ориентированого программирования: по меньшей мере классы 
и инкапсуляцию. Но главным преимуществом этого языка является его строгая типизированность. 
Это облегчает разработку и позволяет писать более читаемый код по сравнению с JavaScript.

На рисунках ~\ref{ris:web_example_1} и ~\ref{ris:web_example_2} видно, приложение состоит из четырех логический частей. 

\begin{figure}[h!]
    \center{\includegraphics[width=0.8\linewidth]{web_example_1}}
    \caption{Верхняя часть веб-сайта}
    \label{ris:web_example_1}
\end{figure}

\begin{figure}[h!]
    \center{\includegraphics[width=0.8\linewidth]{web_example_2}}
    \caption{Нижняя часть веб-сайта}
    \label{ris:web_example_2}
\end{figure}

Первая - шапка сайта. Она содержит название 
приложения и ссылки на три страницы приложения: основную рабочую область, страницу генерации отчетов и страницу со 
справкой.

Вторая логическая часть - это большая недоступная для ввода область TextArea вместе с набором кнопок, 
отвечающих за переключение используемой таблицы. Эта часть нужна для получения данных, которые 
уже содержатся в базе данных для более удобного добавления новых записей в приложение.

Третья часть -- доступная для ввода область TextArea, в которую автоматически загружаются
JSON прототипы для текущего вида входных данных.

Заключительная область необходима для поиска необходимых книг в сервисе Google Books.

\subsection*{Серверная часть}
\addcontentsline{toc}{subsection}{Серверная часть}

Приложение написано на языке Go, его можно разделить на несколько модулей или пакетов, что и было сделано
в исходном коде программы. 

В первую очередь стоит отметить пакет \texttt{fetcher}, который отвечает за
поиск книг в сервисе Google Books. Он формирует запрос с помощью API-ключа и данных,
введенных пользователем, после чего отправляет его и получает ответ, а дальше возвращает его в необходимом
формате. Именно по этой причине для старта приложения необходимо вводить API-ключ.

Далее следует, пожалуй, самая крупная часть данного приложения: работа с базой данных. Как уже было упомянуто
выше, работа с ней ведется с использованием связки из модуля стандартной библиотеки \texttt{database/sql} и
драйвера \texttt{pq}. Пакет \texttt{database/sql} является исключительно интерфейсом, в то время как именно
драйвер реализует всю необходимую логику. Для их взаимодействия нужно добавить в файл оба этих пакета, используя команду
\texttt{import}.

Для начала будет правильным рассмотреть принципы работы с \texttt{database/sql}.
Подключение к СУБД происходит с помощью метода \texttt{sql.Open}.
Данный метод имеет два параметра:

\begin{itemize}
	\item Имя драйвера -- в данном случае это будет postgres.
	\item Конфигурационная строка -- набор пар ключ-значение, определяющее все параметры подключения. Например,
	      подключение к базе данных <<bibtex>> пользователя <<username>> к СУБД, находящейся на порту 8888, будет выглядеть
	      следующим образом: \texttt{"user=username port=8888 dbname=bibtex"}.
\end{itemize}

Метод \texttt{sql.Open} вернет структуру \texttt{sql.DB}, через которую будут происходить
все дальнейшие взаимодействия с СУБД, в частности, внесение и получение данных.

Следующим шагом стоит выполнить метод \texttt{DB.Ping}, который создает или восстанавливает подключение.
Это позволяет удостовериться, что все работает корректно. После этого СУБД готова к использованию в приложении.

В первую очередь при разработке были реализованы методы \texttt{insert} и \texttt{select} для всех таблиц.
Стоит отметить, что приложение рассчитано исключительно на использование из CLI и веб-приложения и не
предусматривает вставки нескольких записей за раз. \texttt{Select} внутри запроса сразу выполняет
необходимые объединения \texttt{join} для того, чтобы результат был максимально удобен для
использования пользователем.

Следующим шагом были разработаны методы для удаления из таблиц. Как уже было сказано выше,
было решено использовать каскадное удаление на стороне сервера. Для этого были использованы
транзакции. В пакете \texttt{database/sql} они представляются структурой \texttt{Tx}.
Если начинать использовать транзакции в удалении, то было бы логично начать использовать их везде, в том числе
и уже реализованных \texttt{insert} и \texttt{select}. Для этого был создан интерфейс языка Go, который можно увидеть в листинге ~\ref{sqlexecutable}.

\begin{lstlisting}[language=bash, caption = {Интерфейс унификации транзакций.}, captionpos=b, label={sqlexecutable}]
type SQLExecutable interface {
	Exec(string, ...interface{}) (sql.Result, error)
	Query(string, ...interface{}) (*sql.Rows, error)
	QueryRow(string, ...interface{}) *sql.Row
}
\end{lstlisting}

Этот интерфейс необходим для унификации используемых методов структур \texttt{sql.DB} и \texttt{sql.Tx}.
С использованием этого интерфейса стало возможно написание метода \texttt{getSQLExecutable}, который
возвращает структуру транзакции, если она была иницилизирована и обычную \texttt{sql.DB} в ином случае.

После этого достаточно все использования \texttt{sql.DB} заменить на метод \texttt{getSQLExecutable}.
Важно отметить, что для использования такого приема не требуется никакого изменения этих структур.
Дело в том, что, в отличии от, например, языка Java, для имплементации интерфейса структуре не
обязательно явно указывать список реализуемых интерфейсов. Компилятор способен сам определять, подходит
ли структура под выбранный интерфейс. В случае неудачи при компиляции будет выведена ошибка.

Теперь можно перейти непосредственно к методам удаления. Принцип их работы прост: в рамках одной
транзакции последовательно удалять все необходимые таблицы. Рассмотрим, например, последовательность действий
при удалении таблицы \texttt{Course}:

\begin{enumerate}
	\item удалить курс и получить список суррогатных ключей затронутых списков литературы,
	\item по суррогатным ключам удалить списки литературы, получить список суррогатных ключей
	      для связи многие-ко-многим с учебниками,
	\item по суррогатным ключам удалить все связи многие-ко-многим.
\end{enumerate}

Аналогично будет происходить со всеми остальными удалениями.

Последней проблемой в этой части приложения стала реализация обновления. Самым логичным поведением тут
является реализация политики <<при \texttt{insert} добавить новую запись, если записи с таким первичным ключом нет, 
обновить в ином случае>>. Это позволяет свести любые обновления к добавлениям, что уменьшает количество
кода и на сервере, и на клиентах. Но возникает проблема с тем, что такая возможность реализована во всех
СУБД немного по разному. Например, в SQLite для таких целей есть отдельная команда \texttt{replace}.
В PostgreSQL есть конструкция <<ON CONFLICT>>. Добавление в таблицу \texttt{Lecturer} с учетом этой конструкции можно
увидеть в листинге ~\ref{insert_lecturer}.

\begin{lstlisting}[language=bash, caption = {SQL запрос добавления записи в таблицу Lecturer}, captionpos=b, label={insert_lecturer}]
INSERT INTO schema.lecturer(
    lecturer_name, 
    lecturer_date_of_birth,
    lecturer_department_id, 
    lecturer_timestamp
    ) 
    VALUES ($1, $2, $3, $4)
        ON CONFLICT(lecturer_name, lecturer_date_of_birth) 
        DO UPDATE SET
            lecturer_department_id=
                EXCLUDED.lecturer_department_id,
            lecturer_timestamp=
                EXCLUDED.lecturer_timestamp,
            lecturer_is_deleted=
                FALSE;
\end{lstlisting}

При отладке этой части программы была обнаружена одна особенность языка Go, которая
сильно усложняет тестирование работоспособности приложения. Дело в том, что тип 
error, являющийся стандартным типом для ошибок, не предоставляет трассировку стека. Другими словами, по ошибке невозможно понять,
в каком месте программы она была вызвана в случае, если обработка этой ошибки делегируется
вызывающему методу. Дело в том, что тип error, которым представляются все ошибки в языке
Go, является всего лишь интерфейсом с одним методом \texttt{Error() string}, возвращающим текст ошибки.
Но, с другой стороны, такой подход позволяет реализовывать собственные типы ошибок,
что и было сделано данном приложении. Ошибка представляется типом \texttt{Error}, который можно увидеть в листинге ~\ref{error}:

\begin{lstlisting}[language=bash, caption = {Стурктура Error.}, captionpos=b, label={error}]
type Error struct {
	Message       string
	StackTrace    string
	DatabaseError *pq.Error
}
\end{lstlisting}

Перечислим назначение каждого поля.

\begin{enumerate}
	\item \texttt{Message} -- текстовое сообщение об ошибке,
	\item \texttt{StackTrace} -- трассировка ошибки, созданная на основе метода \texttt{runtime.Stack} пакета \texttt{runtime},
	\item \texttt{DatabaseError} -- указатель на ошибку драйвера \texttt{pq}, работающего с PostgreSQL. Указатель равен \texttt{nil}
	      в случае, если ошибка произошла не в драйвере.
\end{enumerate}

Теперь следует описать принципы работы программного интерфейса. В разработанном приложении используется HTTP-сервер из стандартной
библиотеки языка Go. Использовать этот пакет очень просто: необходимо добавить необходимый метод с нужной сигнатурой и
URL-строку в сервер с помощью метода \texttt{http.HandleFunc}. В этой сигнатуре функция должна принимать два аргумента:
\texttt{http.ResponseWriter}, который предназначен для отправки ответа клиенту и \texttt{http.Request}, содержащий информацию
о поступившем запросе. В приложении на каждый запрос первым делом выполняется проверка входных данных, если таковые имеются.
По большей части это проверка на наличие тела запроса, если оно должно присутствовать.

Также при обработке каждого запроса в ответ добавляется заголовок \texttt{Access-Control-Allow-Origin}, который необходим
для того, чтобы при использовании этого API браузером все работало корректно. Без этого хэдера браузеры просто блокируют запросы
к серверу. Согласно документации компании <<Mozilla>>\cite{Access-Control-Allow-Origin}, этот заголовок позволяет
установить, с какими сайтами сервер умеет работать, а с какими не умеет. Например, в него можно записать \texttt{https://developer.mozilla.org},
что означает, что сервер умеет работать только с одним этим сайтом. Если значение будет \texttt{*}, тогда сервер будет работать с абсолютно
любым клиентом.

Для каждой таблицы имеется следующий набор методов:

\begin{enumerate}
	\item получение содержимого таблицы,
	\item добавление записи в таблицу,
	\item удаление из таблицы,
	\item получение прототипа JSON-файла для записи таблицы.
\end{enumerate}

Также сервер имеет три дополнительных метода помимо описанной структуры. 
Первый метод выполняет миграцию списка литературы с одного года на другой.
Второй нужен для получения прототипа JSON-файла миграции, а третий выполняет генерацию отчета
и записывает его на клиент в виде текстового файла. В случае веб-версии происходит создание текстового файла в формате \texttt{.bib}, который
сохраняется в загрузки.
\clearpage
% -----------------

% Глава 3
%\section*{Разработка серверного приложения}
\addcontentsline{toc}{section}{Разработка серверного приложения}

Разработанное серверное приложение можно разделить на несколько модулей или пакетов языка Go, что и было сделано
в исходном коде программы. В первую очередь стоит отметить пакет \texttt{fetcher}, который отвечает за
поиск книг в сервисе Google Books. Все, что он делает: формирует запрос с помощью API-ключа и данных,
введенных пользователем, после чего отправляет его и получает ответ, а дальше возвращает его в необходимом
формате. Именно по этой причине для старта приложения необходимо вводить API-ключ.

Далее следует, пожалуй, самая крупная часть данного приложения: работа с базой данных. Как уже было упомянуто
выше, работа с базой данных ведется с использованием связки из модуля стандартной библиотеки \texttt{database/sql} и
драйвера \texttt{pq}. Пакет \texttt{database/sql} является исключительно интерфейсом, в то время как именно
драйвер реализует всю необходимую логику. Для их взаимодействия нужно добавить в файл оба этих пакета, используя команду
\texttt{import}.

Для начала будет правильным рассмотреть принципы работы с \texttt{database/sql}.
Подключение к СУБД происходит с помощью метода \texttt{sql.Open}.
Данный метод имеет два параметра:

\begin{itemize}
    \item имя драйвера -- в данном случае это будет postgres
    \item конфигурационная строка -- набор пар ключ-значение, определяющее все параметры подключения. Например,
    подключение к базе данных <<bibtex>> пользователя <<username>> к СУБД, находящейся на порту 8888, будет выглядеть
    следующим образом: "user=username port=8888 dbname=bibtex"
\end{itemize}

Метод \texttt{sql.Open} вернет структуру \texttt{sql.DB}, через которую будут происходить
все дальнейшие взаимодействия с СУБД, в частности, внесение и получение данных.

Следующим шагом стоит выполнить метод \texttt{DB.Ping}, который создает или восстанавливает подключение.
Это позволяет удостовериться, что все работает корректно. После этого СУБД готова к использованию в приложении.

В первую очередь при разработке были реализованы методы \texttt{insert} и \texttt{select} для всех таблиц.
Стоит отметить, что приложение рассчитано исключительно на использование из CLI и веб-приложения и не
предусматривает вставки нескольких записей за раз. \texttt{Select} внутри запроса сразу выполняет
необходимые объединения \texttt{join} для того, чтобы результат был максимально удобен для
использования пользователем.

Следующим шагом были разработаны методы для удаления из таблиц. Как уже было сказано выше,
было решено использовать каскадное удаление на стороне сервера. Для этого были использованы
транзакции. В пакете \texttt{database/sql} они представляются структурой \texttt{Tx}.
Если при использовать транзакции в удалении, было логично начать использовать их везде, в том числе
и уже реализованных \texttt{insert} и \texttt{select}. Для этого был создан интерфейс языка Go.

Выглядит он следующим образом:

\begin{lstlisting}
type SQLExecutable interface {
	Exec(string, ...interface{}) (sql.Result, error)
	Query(string, ...interface{}) (*sql.Rows, error)
	QueryRow(string, ...interface{}) *sql.Row
}
\end{lstlisting}

Этот интерфейс необходим для унификации используемых методов структур \texttt{sql.DB} и \texttt{sql.Tx}.
С использованием этого интерфейса стало возможно написание метода \texttt{getSQLExecutable}, который
возвращает структуру транзакции, если она была иницилизирована и обычную \texttt{sql.DB} в ином случае.

После этого достаточно все использования \texttt{sql.DB} заменить на метод \texttt{getSQLExecutable}.
Важно отметить, что для использования такого приема не требуется никакого изменения этих структур.
Дело в том, что, в отличии от, например, языка Java, для имплементации интерфейса структуре не
обязательно явно указывать список реализуемых интерфейсов. Компилятор способен сам определять, подходит
ли структура под выбранный интерфейс. В случае неудачи при компиляции будет выведена ошибка.

Теперь можно перейти непосредственно к методам удаления. Принцип их работы прост: в рамках одной
транзакции последовательно удалять все необходимые таблицы. Рассмотрим, например, удаление
таблицы \texttt{Course}.

\begin{enumerate}
    \item Шаг 1. Удалить курс и получить список суррогатных ключей затронутых списков литературы.
    \item Шаг 2. По суррогатным ключам удалить списки литературы, получить список суррогатных ключей
    для связи многие-ко-многим с учебниками.
    \item Шаг 3. По суррогатным ключам удалить все связи многие-ко-многим.
\end{enumerate}

Аналогично будет происходить со всеми остальными удалениями.

Последней проблемой в этой части приложения стала реализация обновления. Самым логичным поведением тут
является реализация политики <<при \texttt{insert} добавить новую запись, если записи с таким первичным ключом нет, 
обновить в ином случае>>. Это позволяет свести любые обновления к добавлениям, что уменьшает количество
кода и на сервере, и на клиентах. Но возникает проблема с тем, что такая возможность реализована во всех
СУБД немного по разному. Например, в SQLite для таких целей есть отдельная команда \texttt{replace}.
В PostgreSQL есть конструкция <<ON CONFLICT>>. С учетом этой конструкции, добавление
в базу данных будет выглядеть следующим образом для таблицы \texttt{Lecturer}:

\begin{lstlisting}
INSERT INTO schema.lecturer(
    lecturer_name, 
    lecturer_date_of_birth,
    lecturer_department_id, 
    lecturer_timestamp
    ) 
    VALUES ($1, $2, $3, $4)
        ON CONFLICT(lecturer_name, lecturer_date_of_birth) 
        DO UPDATE SET
            lecturer_department_id=
                EXCLUDED.lecturer_department_id,
            lecturer_timestamp=
                EXCLUDED.lecturer_timestamp,
            lecturer_is_deleted=
                FALSE;
\end{lstlisting}

При отладке этой части программы была обнаружена одна особенность языка Go, которая
сильно усложняет тестирование работоспособности программы. Дело в том, что ошибки 
error не предоставляют трассировку стека. Другими словами, по ошибке невозможно понять,
в каком месте программы она была вызвана в случае, если обработка этой ошибки делегируется
вызывающему методу. Дело в том, что тип error, которым представляются все ошибки в языке
Go, является всего лишь интерфейсом с одним методом, возвращающим строку ошибки.
Но, с другой стороны, такой подход позволяет реализовывать собственные типы ошибок,
что и было сделано. Ошибка представляется типом \texttt{Error}, имеющим следующуя структуру:

\begin{lstlisting}
type Error struct {
	Message       string
	StackTrace    string
	DatabaseError *pq.Error
}
\end{lstlisting}

Перечислим назначение каждого поля.

\begin{enumerate}
    \item \texttt{Message} -- текстовое сообщение об ошибке,
    \item \texttt{StackTrace} -- трассировка ошибки, созданная на основе метода \texttt{runtime.Stack} пакета \texttt{runtime},
    \item \texttt{DatabaseError} -- указатель на ошибку драйвера \texttt{pq}, работающего с PostgreSQL. Указатель равен \texttt{nil}
        в случае, если ошибка произошла не в драйвере.
\end{enumerate}

Теперь следует описать принципы работы программного интерфейса. В разработанном приложении используется HTTP-сервер из стандартной
библиотеки языка Go. Использовать этот пакет очень просто: необходимо добавить необходимый метод с нужной сигнатурой и
URL-строку в сервер с помощью метода \texttt{http.HandleFunc}. В этой сигнатуре функция должна принимать два аргумента:
\texttt{http.ResponseWriter}, который предназначен для отправки ответа клиенту и \texttt{http.Request}, содержащий информацию
о поступившем запросе. В приложении на каждый запрос первым делом выполняется проверка входных данных, если таковые имеются.
По большей части это проверка на наличие тела запроса, если таковое должно присутствовать.

Также при обработке каждого запроса в ответ добавляется заголовок \texttt{Access-Control-Allow-Origin}, который необходим
для того, чтобы при использовании этого API браузером все работало корректно. Без этого хэдера браузеры просто блокируют запросы
к серверу. Согласно документации компании <<Mozilla>>\cite{Access-Control-Allow-Origin}, этот заголовок позволяет
установить, с какими сайтами сервер умеет работать, а с какими не умеет. Например, в него можно записать \texttt{https://developer.mozilla.org},
что означает, что сервер умеет работать только с одним этим сайтом. Если значение будет \texttt{*}, тогда сервер будет работать с абсолютно
любым клиентом.

Для каждой таблицы имеется следующий набор методов:

\begin{enumerate}
    \item получение содержимого таблицы,
    \item добавление записи в таблицу,
    \item удаление из таблицы,
    \item получение прототипа JSON-файла для записи таблицы.
\end{enumerate}

Также сервер имеет три дополнительных метода помимо описанной структуры. 
Первый метод выполняет миграцию списка литературы с одного года на другой.
Второй нужен для получения прототипа JSON-файла миграции, а третий выполняет генерацию отчета
и записывает его на клиент в виде текстового файла. В случае веб-версии происходит создание текстового файла в формате \texttt{.bib}, который
сохраняется в загрузки.
%\clearpage
% -----------------

% Глава 4
\section*{Тестирование}
\addcontentsline{toc}{section}{Тестирование}

Как уже отмечалось выше, потенциальная производительность данного приложения не является главной характеристикой
разрабатываемого приложения, так как даже потенциально у приложения не будет больше десяти пользователей,
подключенных одновременно. Это означает, что в данном разделе будет более целесообразно остановиться на
рассмотрении того пользовательского опыта, который сможет получить человек при использовании разработанного
приложения. Такое тестирование называется тестированием эргономичности или usability testing.

Итак, основная цель, которую будет приследовать пользователь при использовании приложения -- это создание
текстового файла со списком использованной литературы, который можно импортировать в LaTeX-документ.

Так как в проекте предусмотрено два различных и параллельных друг другу способа взаимодействия с приложением,
ниже будет рассмотрен каждый из них. Для удобства работа с приложением командной строки будет приведена
в листингах, а с веб-версией -- снимками экрана. Причем в листингах знаком ">" будет обозначаться введенная
bash-команда.

Первым делом необходимо создать кафедру, на которой работает пользователь. Для наглядности будет рассмотрена
кафедра <<Теоретическая информатика и компьютерные технологии>>.

Необходимые действия, которые нужно сделать в приложении командной строки, приведены в листинге ~\ref{addDepartment}.

\begin{lstlisting}[language=bash, caption = {Добавление кафедры}, captionpos=b, label={addDepartment}]
> cli department prototype
Open department.txt and fill prototype struct with correct data
> nano department.txt 
> cli department add
\end{lstlisting}

В файле department.txt будет находиться JSON-объект, для которого будет необходимо заполнить поле \texttt{title}
необходимым значением.
\clearpage
% -----------------

% Вывод
\section*{Заключение}
\addcontentsline{toc}{section}{Заключение}

В ходе данной курсовой работы был выполнен анализ и сравнение двух различных ветвей развития
систем управления базами данных: SQL и NoSQL, была произведена разработка серверного REST-приложения
с HTTP-интерфейсом.

Структура разработанного приложения выглядит следующим образом:

\begin{enumerate}
    \item База данных PostreSQL.
    \item Серверное приложение на языке Go с использованием драйвера для PostreSQL.
    \item Веб-сайт на языке TypeScript с использованием фреймворка Bootstrap.
    \item Приложение командной строки на языке Go с использованием фреймворка Cobra.
\end{enumerate}

Таким образом, было разработано приложение, выполняющее хранение и обработку информации о списках
литературы учебных курсов и была продемонстрирована его работоспособность на примере курса под
названием <<Базы данных>>.
\clearpage
% -----------------

% Список литературы
\renewcommand{\bibname}{Список использованной литературы}
\addcontentsline{toc}{section}{Список используемой литературы}
\begin{thebibliography}{1}

% Документация по Qt[Электрон. ресурс] // Режим доступа: http:// doc.qt.io/qt-5/, свободный. — Загл. с экрана.(Дата обращения 05.12.2017)
% 1. Ландау Л.Д., Лифшиц Е.М. Механика. — Издание 4-е, исправленное. — М.: Наука, 1988. — 215с.

\bibitem{RFC2616}
Стандарт RFC 2616 протокола HTTP[Электрон. ресурс] // Режим доступа: https://tools.ietf.org/html/rfc2616, свободный. -- Загл. с экрана.(Дата обращения 16.06.2018)

\bibitem{REST}
Р.Т. Филдинг. Representational state transfer[Электрон. ресурс] // Режим доступа: http://www.ics.uci.edu/\textasciitilde fielding/pubs/dissertation/rest\_arch\_style.htmб
свободный. -- Загл. с экрана.(Дата обращения 22.05.2018).

\bibitem{Web-book}
М. Макдональд. Веб-разработка. Исчерпывающее руководство -- Санкт-Петербург: Питер, 2017. - 638 с.

\bibitem{GettingStartedWithNoSQL}
V. Gaurav. Getting Started with NoSQL - Birmingham: Packt Publishing Ltd, 2013. - 118 с.

\bibitem{MongoDBDocs}
Документация СУБД MongoDB[Электрон. ресурс] // Режим доступа: https://docs.mongodb.com, свободный. -- Загл. с экрана.(Дата обращения 16.06.2018).

\bibitem{TypeScript}
Документация языка программирования TypeScript[Электрон. ресурс] // Режим доступа: https://www.typescriptlang.org/docs/home.html, свободный. 
-- Загл. с экрана.(Дата обращения 16.06.2018).

\bibitem{Access-Control-Allow-Origin}
Документация для заголовка Access-Control-Allow-Origin[Электрон. ресурс] // Режим доступа: https://developer.mozilla.org/en-US/docs/Web/HTTP/Headers/Access-Control-Allow-Origin,
свободный. -- Загл. с экрана.(Дата обращения 22.05.2018).

\end{thebibliography}
% \bibliographystyle{unsrt}
% \bibliography{literature}
% -----------------

\end{document}
